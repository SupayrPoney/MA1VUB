\mychapter{1}{Main concepts}
\section{Clausal Logic}
A logic system consists of:
\begin{enumerate}
\item \textbf{Syntax}: which sentences are legal
\item \textbf{Semantics}: what is the truth value of a sentence
\item \textbf{Proof theory}: how to derive new sentences (theorems) from assumed ones (axioms) by means of inference rules.
\end{enumerate}
A logic system should be:
\begin{enumerate}
\item \textbf{Sound}: anything you can prove is true.
\item \textbf{Complete}: anything true can be proven.
\end{enumerate}

There are different clausal logic systems (in order of increasing expressiveness):
\begin{enumerate}
\item \textbf{Propositional clausal logic}:

Only composed by atoms with are statements with a value of true or false.

e.g.: married; bachelor :- man, adult.
\item \textbf{Relational clausal logic}:

Introduces relations between these atoms (constants or variables).

e.g.: likes(Declarative, S) :- crazy(S).
\item \textbf{Full clausal logic}:

To avoid explicit listing of clauses, we introduce function symbols and complex terms (\textit{functors}) 

e.g.: loves(X, person\_loved\_by\_(X)).
\item \textbf{Definite clausal logic}:

This is what Prolog uses. Clauses only have one true litteral.

e.g.: A :- $B_1$ , ... $B_n$

\end{enumerate}
\section{Cut}

\begin{displayquote}
\textbf{
Once you reached me, stick with all variable substitutions you've found after you entered my clause.}
\end{displayquote}
\section{Graphs}
\section{Backward chaining \& Forward chaining}
\section{Definite clause grammar}
\section{Reasoning}
\subsection{Default Reasoning}
\subsection{Abductive Reasoning}
\subsection{Inductive Reasoning}
\section{Program completion}
\section{Closed world assumption}
\begin{displayquote}
\textbf{
Everything that is not known to be true is false.}
\end{displayquote}
\section{Bottom up induction \& Top down induction}
